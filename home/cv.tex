% LaTeX resume using res.cls
\documentclass[margin]{res}
%\usepackage{helvetica} % uses helvetica postscript font (download helvetica.sty)
%\usepackage{newcent}   % uses new century schoolbook postscript font 
\setlength{\textwidth}{5.1in} % set width of text portion
\usepackage{hyperref}

\begin{document}

% Center the name over the entire width of resume:
 \moveleft.5\hoffset\centerline{\large\bf STAVROS TSOGKAS}
% Draw a horizontal line the whole width of resume:
 \moveleft\hoffset\vbox{\hrule width\resumewidth height 1pt}\smallskip
% address begins here
% Again, the address lines must be centered over entire width of resume:
 \moveleft.5\hoffset\centerline{6 King's College Rd., Toronto, Ontario, Canada}
 \moveleft.5\hoffset\centerline{\url{http://www.cs.toronto.edu/~tsogkas}}
 \moveleft.5\hoffset\centerline{email: tsogkas@cs.toronto.edu}


\begin{resume}
 
\section{Research \\ Interests} 
My research interests are in the broad areas of 
computer vision and machine learning, with a focus on deep learning. 
I am particularly interested in the use of mid-level representations to bridge the gap
between bottom-up and top-down processing and solve problems such as object detection,
segmentation and grouping. 
I have devoted a large part of my research on recovering such representations, medial axes and object parts in particular, in natural and medical images.

\section{Education} 
\textbf{CentraleSup\`elec} \hfill Jan. 2016\\
Ph.D. in Mathematics and Computer Science\\
Thesis: Mid-level Representations for Modeling Objects\\
Advisor: Iasonas Kokkinos 

\textbf{National Technical University of Athens} \hfill Sep. 2011 \\  
Diploma in Electrical and Computer Engineering\\
Thesis: Learning-Based Symmetry Detection in Natural Images\\
Advisors: Petros Maragos, Iasonas Kokkinos
 
\section{Peer-reviewed Conference Publications} 
	\begin{itemize}
		\item AMAT: Medial Axis Transform for Natural Images, \emph{ICCV 2017}\\
			\textbf{S. Tsogkas}, S. Dickinson 
		\item  Prior-based Coregistration and Cosegmentation, \emph{MICCAI 2016}\\
			M. Shakeri*, E. Ferrante*, \textbf{S. Tsogkas}, S. Lippe, S. Kadoury, I. Kokkinos,  N. Paragios (* denotes equal contribution)   
		\item  Subcortical Brain Structure Segmentation Using FCNNs, \emph{ISBI 2016} \textbf{(oral)}\\
			\textbf{S. Tsogkas*}, M. Shakeri*, E. Ferrante, S. Lippe, S. Kadoury, N. Paragios, I. Kokkinos (* denotes equal contribution)   
		\item   Accurate Human-Limb Segmentation in RGB-D images for Intelligent Mobility Assistance Robots\\
			\emph{ICCV 2015 $3^{rd}$ Workshop on Assistive Computer Vision and Robotics}\\
			S. Chandra, \textbf{S. Tsogkas}, I. Kokkinos
		\item  Deformable Part Models with CNN Features,\\ \emph{ECCV 2014 Parts and Attributes workshop}\\
		P.-A. Savalle, \textbf{S. Tsogkas}, G. Papandreou and I. Kokkinos 
		\item  Superpixel-grounded Deformable Part Models, \emph{CVPR 2014}\\
			E. Trulls, \textbf{S. Tsogkas}, I. Kokkinos, A. Sanfeliu, F.Moreno
		\item Understanding Objects in Detail with Fine-grained Attributes, \emph{CVPR 2014} \\
		A. Vedaldi, S. Mahendran, \textbf{S. Tsogkas}, S. Maji, B. Girshick, J. Kannala, E. Rahtu, I. Kokkinos, M. B. Blaschko, D. Weiss, B. Taskar, K. Simonyan, N. Saphra, S. Mohamed 
		\item Learning-Based Symmetry Detection in Natural Images, \emph{ECCV 2012} \\
			\textbf{S. Tsogkas}, I. Kokkinos
	\end{itemize}

\section{Reports}
	\begin{itemize}
		\item ICCV 2017 Challenge: Detecting Symmetry in the Wild (editorial), \\
		\emph{Detecting symmetry in the wild workshop, ICCV 2017}\\
		Chris Funk*, Seungkyu Lee*, Martin R. Oswald*, \textbf{Stavros Tsogkas*}, 
		Wei Shen, Andrea Cohen, Sven Dickinson, Yanxi Liu 
		(* denotes equal contribution)
		\item  Deep Learning for Semantic Part Segmentation with High-Level Guidance, \emph{arXiv report}\\
		\textbf{S. Tsogkas}, I. Kokkinos, G. Papandreou, A. Vedaldi 
  \end{itemize}

\section{Research Experience} 
\textbf{University of Toronto} \hfill Oct. 2016 - present\\
Postdoctoral fellow, Computer Science Department\\
Supervisor: Sven Dickinson

\textbf{CentraleSup\`elec} \hfill Jan. 2016 - Aug. 2016\\
Research engineer, CVN lab \\
Supervisor: Nikos Paragios \\
\emph{Convolutional neural networks for semantic 
	segmentation of organs in computed tomography scans.
}

\textbf{Oxford University (Visual Geometry Group)} \hfill Aug.-Nov. 2014\\
Research intern \\
Supervisor: Andrea Vedaldi. \\
\emph{Combined convolutional neural networks 
	and restricted boltzmann machines for semantic segmentation 
	of object parts.
}


\section{Teaching Experience} 
	\textbf{Teaching assistant (CentraleSup\`elec)}   \hfill 2011-2015 
	\begin{itemize}
		\item Signal Processing  (undegrad course).  
		\item Computer Vision  (undegrad course).
		\item Machine Learning for Computer Vision (MVA master course)
	\end{itemize}
	\textbf{Invited lecturer (CentraleSup\`elec/ESSEC)} \hfill 2016 \\
	MSc in Data Science and Business Analytics \\
	Seminar on deep learning theory and tools.


\section{Professional Activities} 
	\textbf{Reviewer}, IEEE TPAMI, IJCV, CVIU, IMAVIS, IEEE ICCV, IEEE CVPR, ECCV, ICVGIP,
	Morgan \& Claypool Synthesis lectures on Computer Vision \\
	\textbf{Co-organizer} of the ``Detecting Symmetry in the Wild'' workshop, \\
	in conjunction with ICCV 2017, Venice, Italy. \\
	\textbf{Treasurer}, IEEE NTUA Student Branch \hfill 2010-2011 \\
	\textbf{Chairman}, IEEE NTUA Student Branch \hfill 2011-2012 \\
	\textbf{Student member} IEEE \hfill 2012-2015

\section{Programming Skills} MATLAB, C/C++, Lua, Latex, Caffe, MatConvNet, Torch. 
\newline

\section{Distinctions} Outstanding reviewer award (ECCV 2016)
\section{Citizenship} Greek
\section{Languages} English (fluent), French (proficient), Greek (native).
 
\end{resume}
\end{document}







