% LaTeX resume using res.cls
\documentclass[margin]{res}
%\usepackage{helvetica} % uses helvetica postscript font (download helvetica.sty)
%\usepackage{newcent}   % uses new century schoolbook postscript font 
\setlength{\textwidth}{5.1in} % set width of text portion
\usepackage{hyperref}

\begin{document}

% Center the name over the entire width of resume:
 \moveleft.5\hoffset\centerline{\large\bf STAVROS TSOGKAS}
% Draw a horizontal line the whole width of resume:
 \moveleft\hoffset\vbox{\hrule width\resumewidth height 1pt}\smallskip
% address begins here
% Again, the address lines must be centered over entire width of resume:
 \moveleft.5\hoffset\centerline{92295 Grande Voie des Vignes, Ch\^atenay-Malabry, France}
 \moveleft.5\hoffset\centerline{\url{http://cvn.ecp.fr/personnel/tsogkas/}}
 \moveleft.5\hoffset\centerline{email: stavros.tsogkas@centralesupelec.fr}


\begin{resume}
 
\section{RESEARCH INTERESTS} Computer vision (shape analysis, semantic segmentation, 
fine-grained recognition, object detection), deep learning.
\newline

\section{EDUCATION} 
\textbf{CentraleSup\'elec} \hfill Jan. 2016\\
PhD on Computer Vision\\
Thesis: Mid-level Representations for Modeling Objects\\
Advisor: Iasonas Kokkinos 

\textbf{National Technical University of Athens} \hfill Sep. 2011 \\  
Diploma in Electrical and Computing Engineering\\
Thesis: Learning-Based Symmetry Detection in Natural Images\\
Advisors: Petros Maragos, Iasonas Kokkinos
 
\section{INTERNSHIPS} Research intern at Oxford University (Visual Geometry Group) \hfill Aug.-Nov. 2014\\
Project: Semantic segmentation of object parts. \\ Supervisor: Andrea Vedaldi.

\section{PUBLICATIONS} 
	\begin{itemize}
	\item  Sub-cortical Brain Structure Segmentation Using F-CNNs, \emph{ISBI 2016}\\
		  \textbf{M. Shakeri*}, \textbf{S. Tsogkas*}, E. Ferrante, S. Lippe, S. Kadoury, N. Paragios, I. Kokkinos (* denotes equal contribution)   
	\item   Accurate Human-Limb Segmentation in RGB-D images for Intelligent Mobility Assistance Robots\\
		   \emph{ICCV 2015 $3^{rd}$ Workshop on Assistive Computer Vision and Robotics}\\
		  S. Chandra, \textbf{S. Tsogkas}, I. Kokkinos
	\item   Semantic Part Segmentation with Deep Learning, \emph{arXiv report (under review)}\\
		  \textbf{S. Tsogkas}, I. Kokkinos, G. Papandreou, A. Vedaldi 
	\item  Deformable Part Models with CNN Features,\\ \emph{ECCV 2014 Parts and Attributes workshop}\\
	  P.-A. Savalle, \textbf{S. Tsogkas}, G. Papandreou and I. Kokkinos 
	\item  Superpixel-grounded Deformable Part Models, \emph{CVPR 2014}\\
		  E. Trulls, \textbf{S. Tsogkas}, I. Kokkinos, A. Sanfeliu, F.Moreno
	\item Understanding Objects in Detail with Fine-grained Attributes, \emph{CVPR 2014} \\
	A. Vedaldi, S. Mahendran, \textbf{S. Tsogkas}, S. Maji, B. Girshick, J. Kannala, E. Rahtu, I. Kokkinos, M. B. Blaschko, D. Weiss, B. Taskar, K. Simonyan, N. Saphra, S. Mohamed 
	\item Learning-Based Symmetry Detection in Natural Images, \emph{ECCV 2012} \\
		  \textbf{S. Tsogkas}, I. Kokkinos
	\end{itemize}

\section{SKILLS} MATLAB, C/C++, Latex, Caffe, MatConvNet. 
\newline

\section{TEACHING} Teaching assistant in "Computer Vision" and "Signal Processing" \hfill ECP, 2011-2012 \\(taught by Iasonas Kokkinos)  	
\section{PROFESSIONAL SERVICE} Reviewer for: TPAMI, CVIU, IMAVIS, ICCV, CVPR, ICVGIP. 
\newline

\section{IEEE}
		Treasurer, IEEE NTUA Student Branch \hfill 2010-2011 \\
	    Chairman, IEEE NTUA Student Branch \hfill 2011-2012 \\
	    IEEE Student member	\hfill 2012-2015
 
 \section{LANGUAGES} English (fluent), French (proficient), Greek (native).
 \section{INTERESTS} Piano, bass guitar, poker, board games, travelling.
 
  \section{RESEARCH STATEMENT} My PhD focuses on exploiting mid-level representations for object recognition. These representations provide a middle ground between bottom-up and top-down processing, are more robust than pixel-level features and can be shared among different object classes. The mid-level constructs we consider in the works described below are medial axes, object parts, and convolutional features.
 
In~[1] we successfully transferred ideas from learning-based boundary detection to the detection of medial axes in RGB images. Our method is still the state-of-the-art for this task, and we have been examining a variation based on random forests, that cuts down the computation cost dramatically. We have also improved deformable part models (DPMs) using bottom-up segmentation information. In~[2] we employed soft segmentation masks to ``clean up'' HOG features, separating them into foreground and background components; training DPMs with these enhanced features leads to better detection performance. In~[3] we showed that the prediction of certain attributes can benefit substantially from accurate part detection. In that work I proposed and implemented a coarse-to-fine approach using DPMs, that speeds up part detection 4-7 times, with negligible performance loss.

During the last year I have been focusing on deep learning for object detection and fine-grained semantic segmentation. In~[4], we integrated CNN features in the DPM pipeline obtaining a $14.5\%$ absolute gain in mean average precision (mAP) on the PASCAL VOC 2007 dataset, while keeping computation time low, despite the increased feature dimensionality. In my latest project I investigate the use of CNNs for semantic segmentation of object parts~[5]. Specifically, we use a combination of fully convolutional CNNs and fully-connected CRFs to achieve state-of-the-art performance for part segmentation of faces and pedestrians. On top of this combination we train a Restricted Boltzmann Machine and use it as a shape prior, to guarantee plausible part segmentations.

We have also used CNN features to successfully tackle semantic part segmentation using diverse type of input data and architectures. In~[6] we perform human limb segmentation exploiting both RGB and depth maps (captured from a Kinect sensor) in a single training and testing framework; the long-term goal is to use this framework as part of a complete robotic platform for mobility assistance of elderly people. Finally, in~[7] we design a new architecture for the segmentation of sub-cortical structures in 3D brain MRI volumes, and combine it with a MRF to enforce volumetric homogeneity, achieving state-of-the-art performance in two brain segmentation datasets.  

\begin{thebibliography}{7}
\bibitem[1]{tsogkas2012learning} S. Tsogkas and I. Kokkinos. Learning-based symmetry detection in natural images, \emph{ECCV 2012}.

\bibitem[2]{trulls2014segmentation}  E. Trulls, S. Tsogkas, I. Kokkinos, A. Sanfeliu, and F. Moreno-Noguer.
Segmentation-aware deformable part models, \emph{CVPR 2014}.

\bibitem[3]{vedaldi2014understanding} A. Vedaldi, S. Mahendran, S. Tsogkas, S. Maji, B. Girshick, J. Kannala,
E. Rahtu, I. Kokkinos, M. B. Blaschko, D. Weiss, B. Taskar, K. Simonyan,
N. Saphra, and S. Mohamed. Understanding objects in detail with fine-
grained attributes, \emph{CVPR 2014}.

\bibitem[4]{savalle2014deformable} P.-A. Savalle, S. Tsogkas, G. Papandreou, and I. Kokkinos. Deformable
part models with cnn features, \emph{ECCV, Parts and Attributes Workshop
 (ECCVW 2014)}.

\bibitem[5]{tsogkas2015semantic} S. Tsogkas, I. Kokkinos, G. Papandreou, and A. Vedaldi. Semantic part
segmentation with deep learning, \emph{arXiv preprint arXiv:1505.02438, 2015}.

\bibitem[6]{chandra2015accurate} S. Chandra, S. Tsogkas, I. Kokkinos. Accurate Human-Limb Segmentation in RGB-D images for Intelligent Mobility Assistance Robots, \emph{ICCV 2015 $3^{rd}$ Workshop on Assistive Computer Vision and Robotics}.

\bibitem[7]{shakeri2015subcortical} M. Shakeri*, S. Tsogkas*, E. Ferrante, S. Lippe, S. Kadoury, N. Paragios, I. Kokkinos, Sub-cortical Brain Structure Segmentation Using F-CNNs, \emph{ISBI 2016}.

\end{thebibliography}
 
\end{resume}
\end{document}







